\section{Introduction}

This paper proposes removing constraints on which members may be used in
preconditions and postconditions of member functions. As an example, we propose
that public member functions may use in their preconditions and postconditions
protected and private members. This would lead to less boilerplate code.

As an example, consider the following class.

\begin{lstlisting}
class X {
public:
  //...
  // ptr_ must be non-null
  double get_value() const { return *ptr_; }
private:
  double * ptr_; 
};
\end{lstlisting}

If we want to express the precondition with contracts (and make it available to
callers) we would not be able to write that contract without adding an
additional function to the class to get access to the pointer.

\begin{lstlisting}
class X {
public:
  //...
  double * get_ptr() const { return ptr_; }

  // ptr_ must be non-null
  double get_value() [[expects: get_ptr()!=nullptr]] { return *ptr_; }
private:
  double * ptr_; 
};
\end{lstlisting}

By doing so we are exposing now the \cppid{ptr\_} member to the caller, which is what we
wanted to avoid by encapsulating the pointer in class \cppid{X}.

One could argue that a better solution would be to put the predicate in a public
member function and avoiding to to expose the pointer to the callers.

\begin{lstlisting}
class X {
public:
  //...
  bool valid_ptr() const { return ptr_!=nullptr; }

  // ptr_ must be non-null
  double get_value() [[expects: valid_ptr()]] { return *ptr_; }
private:
  double * ptr_; 
};
\end{lstlisting}

Even if this is slightly better, we still have the problem that the interface
needs to be enlarged just to be able to express the contracts.

In contrast, we propose that the private member function can be used in public
contracts.

\begin{lstlisting}
class X {
public:
  //...

  // ptr_ must be non-null
  double get_value() [[expects: ptr_!=nullptr]] { return *ptr_; }
private:
  double * ptr_; 
};
\end{lstlisting}

