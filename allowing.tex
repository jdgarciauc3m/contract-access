\section{Reasons to simplify control access}

\subsection{Zero-overhead principle}

Private access can help allow contracts to promote zero-overhead
don’t-pay-for-what-you-don’t-use code, while giving users a debugging aid to
find out why they ran into undefined behavior when they do.  Some APIs require
operations to be done in certain order, and contracts could be used to check
for that.  

Consider a \cppid{FastFileChannel} where the caller is required to call open
before calling \cppid{insert\_data}:

\begin{lstlisting}
class FastFileChannel {
public:
  bool open(size_t howbig);  // can fail

  void insert_data(Data data) [[expects: f.is_open() && data.size() <= f.available_size()]];

private:
  File f;
};
\end{lstlisting}

Or a \cppid{FastBuffer} where the caller is required to reserve sufficient
capacity before \cppid{push\_back} is called. In this example we assume that
vector has \cppid{reserve\_nothrow} and \cppid{push\_back\_unchecked} as
proposed in P0132~\cite{p0132}

\begin{lstlisting}

class FastBuffer {
public:
  bool reserve(size_t howmany);

  void push_back() [[expects: buf.size() < buf.capacity() ]];

private:
  vector<char> buf;
};
\end{lstlisting}

Calling \cppid{FastBuffer::push\_back} and
\cppid{FastFileChannel::insert\_data} will perform no capacity checks, and the
\cppid{insert\_data} operation will perform no checks on whether the file is
open for writing or whether the data fits. The intent is that the API makes it
impossible to write slow code: \cppid{file.is\_open()} is deliberately not
exposed, because exposing it has a time and space cost that is not otherwise
incurred.

If the concern is about \emph{``exposing implementation details to the client''}, then
allowing private member access from preconditions and postconditions
of public member functions does not do that, whereas the suggestion to just
write public getters for those privates would expose them.

In summary, the desire is that:

\begin{itemize}

\item Callers are required/assumed to be correct.

\item If the contract is violated, undefined behavior is okay.

\item The programmer can guarantee zero checking overhead of the contract in
release builds. \cppid{File::available\_size()} is assumed to be expensive and the
desire it that it not be callable in non-checked modes, or alternatively it
does nothing. (This further means that such queries are ill-suited to be
exposed in an API.) 

\end{itemize}

There are different ways to design such APIs (have a \cppid{File} create a
\cppid{FastFileChannel} that is guaranteed to be open and sized), but it’s not
obvious that such approaches are better, and they might not be feasible
considering existing code and existing uses.

\subsection{Interface design}

Tony Van Eerd provided an interesting example with \emph{iterators}:

Consider the \cppid{erase()} member function from a container. A precondition
would be that the passed iterator refers to an element in that container.

\begin{lstlisting}
void vector::erase(iterator it) [[expects: is_my_iterator(it)]];
\end{lstlisting}

Function \cppid{is\_my\_iterator()} is only mildly useful outside a contract.
Yet, with current control access it would be necessary to expose it in the
public interfaces. Usually, such public member function could be considered a
code-smell in normal code.

Similarly, consider the sort algorithm where \cppid{last} iterator must be
reachable from \cppid{first}.

\begin{lstlisting}
template <typename I>
void sort(I first, I last) [[expects: reachable(first,last)]];
\end{lstlisting}

Function \cppid{reachable()} might only be an approximation (e.g. are both
iterators in the same container?). Such approximation would not be suitable as a
public member function unless naming has been performed very very carefully.

\subsection{Information hiding}

Access to non exposed members is reasonable because there may already be
suitable private members and they are private for a reason, and exposing them
conflicts with that reason.  Additionally, some such private members may be
conditionally available, and are thus ill-suited for exposure to users.

There are additional observations to be considered. Those observations happen
to match some age-old principles rather well:

\begin{enumerate}

\item If the concern is abuse (\emph{``unmaintainable mini-programs''}), that
should not prevent allowing programmers who know what they are doing from doing
so. This is, \emph{``trust the programmer''}.

\item If a programmatic API does not expose certain members, that is for a
reason. Having to expose those members to use them in declaration-level
contract predicates violates \emph{``do not pay for what you don't use''}, not
just on the performance level, but on that level as well.

\item Finally, being able to keep members private and being able to use them in
declaration-level contract predicates supports programming with the
"zero-overhead principle".

\end{enumerate}

The members may be conditionally available, or may have do-nothing semantics in
release mode, which makes them (in some designs, not all) ill-suited for
exposure to public clients. In addition, using such members is not necessary to
write correct programs, and using them introduces branches, conditions and
checks that, in some cases, the API designer does not want to encourage.

