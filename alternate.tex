\section{Alternate solutions}

It has been suggested by John Lakos to allow using access to private members in function
declarations different from the first one.

\begin{lstlisting}
class X {
private:
  data_store s;
public:
  void insert_data(const D & d);
};

// ### Implementation only below this line
void X::insert_data(const D & d) [[expects: s.is_open()]];
\end{lstlisting}

In such case, access to private members would be allowed only in subsequent
declarations of a function, while the first declaration would be reserved only
for use of public members.

However, this (as pointed out by Ville Voutilainen) is a change that modifies the
current requirement that a \emph{``contract predicate must appear in the first
declaration of a function''} into the new requirement that a \emph{``contract
predicate can appear in a non-first declaration of a function''}. This change
would allow to have multiple non-first declarations with conflicting contract
predicates.

If declarations with conflicting contract predicates are allowed two options are
possible:

\begin{enumerate}

\item Allow conflicting declarations as long as they appear in different
translation units. In that case it would be possible that the same function is
checked in different ways in different translation units. This approach presents
a number of implementation problems. For example, it would be problematic to use
the double entry point technique.

\item Consider that conflicting contract predicates are \emph{ill-formed with no
diagnostic required}. This option is more aligned with the approach alread taken
for inline functions.

\end{enumerate}

Even if option 2 is more desirable than option 1, still this would be another
introduction of \emph{ill-formed/no diagnostic required} which seems an
unnecessary complication.

