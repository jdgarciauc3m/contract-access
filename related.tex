\section{Related work}

Other programming languages have considered this issue. 
In this section we review how access to private members is handled
in some languages supporting contracts.

\subsection{D}

The D programming language allows \emph{``arbitrary D code in the \textbf{in}
and \textbf{out} contract blocks''}. Consequently, it does not impose any
constraint accessibility. In particular use of private members is allowed.

The following example (provided by Throsten Ottosen) is valid in D:

\begin{lstlisting}[morekeywords={in,body}]
class Foo
{
    private int i = 0;
    
    public void foo( int x ) 
    in
    {
        assert( x > 0 );
        assert( i > 0 );
    }
    body
    {
        writeln("doh!"); 
    }
}
\end{lstlisting}

\subsection{Eiffel}

Eiffel offers support for checking preconditions and postconditions as well as
type invariants. Preconditions can be express using only what the caller can
see. However, postconditions my access internal representation as they belong to
the implementation. Finally, invariants may also access internal representation.

\subsection{SPARK}

SPARK does not include runtime checking although it offers preconditions,
postconditions and invariants as a why to support static analysis. Visibility in
SPARK is not considered at type level but at the package level.


