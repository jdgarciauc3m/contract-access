\section{Contracts in other programming languages}

Other programming languages have considered this issue. 
In this section we review how access to private members is handled
in some languages supporting contracts.

\subsection{D}

The D programming language (see \url{https://dlang.org/spec/spec.html}) allows
\emph{``arbitrary D code in the \textbf{in} and \textbf{out} contract
blocks''}. Consequently, it does not impose any constraint accessibility. In
particular use of private members is allowed.

The following example (provided by Thorsten Ottosen) is valid in D:

\begin{lstlisting}[morekeywords={in,body}]
class Foo
{
    private int i = 0;
    
    public void foo( int x ) 
    in
    {
        assert( x > 0 );
        assert( i > 0 );
    }
    body
    {
        writeln("doh!"); 
    }
}
\end{lstlisting}

\subsection{Eiffel}

Eiffel~\cite{iso-eiffel,ecma-eiffel} offers support for checking preconditions
and postconditions as well as type invariants. Preconditions can be expressed
using only what the caller can see. However, postconditions my access internal
representation as they belong to the implementation. Finally, invariants may
also access internal representation.

\subsection{SPARK}

SPARK (see \url{https://www.adacore.com/sparkpro}) is a subset of Ada
2012~\cite{Ada2012}. It does not include runtime checking although it offers
preconditions, postconditions and invariants as a way to support static
analysis. Accessibility in SPARK is not considered at type level but at the
package level.


