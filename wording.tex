\section{Proposed workding}

In section \textbf{[dcl.attr.contract.cond]/4}, edit as follows:

The predicate of a contract condition has the same semantic restrictions as if
it appeared as the first expression-statement in the body of the function it
applies to. Additional access restrictions apply to names appearing in a
contract condition of a member function of class C:

\begin{itemize}

\item Friendship is not considered.

\item \st{For a contract condition of a public member function, no member
of C or of an enclosing class of C is
accessible unless it is a public member of C, or a member of a base
class accessible as a public member
of C (10.8.2).}

\item \st{For a contract condition of a protected member function, no member
of C or of an enclosing class of C
is accessible unless it is a public or protected member of C, or a
member of a base class accessible as a
public or protected member of C.}

\end{itemize}

For names appearing in a contract condition of a non-member function,
friendship is not considered. \emph{[Example:}

\begin{lstlisting}[escapechar=@]
class X {
public:
  int v() const;
  void f() [[expects: x > 0]];               // @\hl{OK}\st{error: x is private}@
  void g() [[expects: v() > 0]];             // OK
  friend void r(int z) [[expects: z > 0]];   // OK
  friend void s(int z) [[expects: z > x]];   // @\hl{OK}\st{error: x is private}@
protected:
  int w();
  void h() [[expects: x > 0]];               // @\hl{OK}\st{error: x is private}@
  void i() [[ensures: y > 0]];               // OK
  void j() [[ensures: w() > 0]];             // OK
  int y;
private:
  void k() [[expects: x > 0]];               // OK
  int x;
};

class Y : public X {
public:
  void a() [[expects: v() > 0]];             // OK
  void b() [[ensures: w() > 0]];             // @\hl{OK}\st{error: w is protected}@
protected:
  void c() [[expects: w() > 0]];             // OK
};

\end{lstlisting}

\emph{end example]}

