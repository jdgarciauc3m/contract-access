\section{Problems with the current status quo}

Currently the rules defining what can be used in a precondition and a
postcondition are governed by clause \textbf{[dcl.attr.contract.cond]}.
In summary those rules are:

\begin{itemize}

\item The predicate has the same semantic restrictions as if it appeared as the
first statement in the body of the function.

\item A contract condition of a public member function can only use public
members or members of base classes that can be used as public members.

\item A contract condition of a protected member function can only use public or
protected members or members of base classes that can be used as public or
protected members.

\end{itemize}

This leads to a number of problematic situations.

\subsection{Exposing implementation details in the interface}

On motivating case discussed at WG21 mailing list was:

\begin{lstlisting}
class X {
  private:
    internal_store s;
  public:
    void open();
    void insert_data(const D & data) [[expects: s.is_open()]];
};

//...

void insert_my_data(X & x, const D & d1, const D & d2) {
  x.insert_data(d1);
  x.insert_data(d2);
}
\end{lstlisting}

The obvious precondition for function \cppid{insert\_my\_data()} is that
\cppid{x.open()} must be called before. However, we cannot write that if we
cannot used the private data member \cppid{s}.

Peter Dimov suggested, three choices for the precondition of
\cppid{insert\_data()} that we study here:

\begin{enumerate}

\item \cppid{X::insert\_data()} can state the preconditions in the form of
\cppid{[[}\cppkey{expects}\cppid{]]} being able to access private members. In
that case, the author of \cppid{insert\_my\_data()} cannot express its own
precondition.

\item \cppid{X::insert\_data()} cannot use private members in its
\cppid{[[}\cppkey{expects}\cppid{]]} preconditions but can use an
\cppid{}\cppkey{assert}\cppid{]]} in its body. In that case, the author of
\cppid{insert\_my\_data()} cannot express its own preconditions. Consequently,
this case is equivalent to the previous one.

\item \cppid{X} could make its private state to become public, but this would
lead to making the internal state of \cppid{X} visible to its clients.

\end{enumerate}

From these three scenarios, the worst case is scenario 3 because it goes against
abstraction and would favor a style where many implementation details are
visible to clients. The key difference between options 1 and 2 are the ability
(in option 1) to use private members in public contracts. However, a drawback of
option 2 is that the precondition is no longer visible to the compiler at the
caller site which reduces its usefulness.

\subsection{Subverting the rules with inheritance}

If a given class has a private member it will not be possible to use that member
in the preconditions of that class. However, a derived class that exposes the
member through \cppkey{using} would be able to use it in its own preconditions.

Additionally a base class with a private virtual member function may use private
members in its contract conditions. If a derived class overrides the member
function as public we face the paradox that the overridden function needs to have
the same contract conditions but that those conditions cannot be used because
they cannot make use of private members.

\subsection{Violation of information hiding principle}

Some have argued that using private members in the contract condition of a
public function would be leaking implementation details in the interface.
However, it should be noted that we usually have a number of implementation
details in header files today (inline function bodies, private members of
classes or whole implementation of templates).

The rationale behind information hiding principle is to limit the number of
operations that have access to implementation details and to guarantee that
invariants are kept. It should be noted that by allowing the use of private
members in preconditions and postconditions we are not changing anything about
this.

Firstly, the invariants will be still held. This is a consequence from the fact
that any precondition or postcondition that performs and observable modification
of the state of the class would lead to undefined behavior.

Secondly, the client code cannot gain access to private members through the
preconditions or postconditions. The only thing that client code may know is
whether contract conditions are satisfied or not.

\subsection{How do we handle the problems?}

It is true that in most of the cases preconditions and postconditions should be
expressed in terms of the public interface. However there cases that making of
this an absolute rule would introduce excessive complications. On the other hand
allowing access to private members are a simple solution for those cases.


