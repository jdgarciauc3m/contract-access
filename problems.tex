\section{Problems with the current status quo}

Currently the rules defining what can be used in a precondition and a
postcondition are governed by clause \textbf{[dcl.attr.contract.cond]}.
In summary those rules are:

\begin{itemize}

\item The predicate has the same semantic restrictions as if it appeared as the
first statement in the body of the function.

\item A contract condition of a public member function can only use public
members or members of base classes that can be used as public members.

\item A contract condition of a protected member function can only use public or
protected members or members of base classes that can be used as public or
protected members.

\end{itemize}

This leads to a number of problematic situations.

\subsection{Exposing implementation details in the interface}

One motivating case discussed at WG21 mailing list was:

\begin{lstlisting}
class X {
  private:
    internal_store s;
  public:
    void open();
    void insert_data(const D & data) [[expects: s.is_open()]];
};

//...

void insert_my_data(X & x, const D & d1, const D & d2) {
  x.insert_data(d1);
  x.insert_data(d2);
}
\end{lstlisting}

The obvious precondition for function \cppid{insert\_my\_data()} is that
\cppid{x.open()} must be called before. However, we cannot write that if we
cannot use the private data member \cppid{s}.

Peter Dimov suggested, three choices for the precondition of
\cppid{insert\_data()} that we study here:

\begin{enumerate}

\item \cppid{X::insert\_data()} can state the preconditions in the form of
\cppid{[[}\cppkey{expects}\cppid{]]} being able to access private members. In
that case, the author of \cppid{insert\_my\_data()} cannot express its own
precondition.

\item \cppid{X::insert\_data()} cannot use private members in its
\cppid{[[}\cppkey{expects}\cppid{]]} preconditions but can use an
\cppid{}\cppkey{assert}\cppid{]]} in its body. In that case, the author of
\cppid{insert\_my\_data()} cannot express its own preconditions. Consequently,
this case is equivalent to the previous one.

\item \cppid{X} could make its private state to become public, but this would
lead to making the internal state of \cppid{X} visible to its clients.

\end{enumerate}

From these three scenarios, the worst case is scenario 3 because it goes against
abstraction and would favor a style where many implementation details are
visible to clients. The key difference between options 1 and 2 are the ability
(in option 1) to use private members in public contracts. However, a drawback of
option 2 is that the precondition is no longer visible to the compiler at the
caller site which reduces its usefulness.

\subsection{Ensuring single initialization}

Another example (provided by Lo\"{i}c Joly) where access control in contracts leads excesive implementation
leakage to clientes is single initialization of global state.

\begin{lstlisting}
class System {
public:
  static void globalInit()
    [[expects: !alreadyCalled]]
    [[ensures: alreadyCalled]];
private:
  static inline bool alreadyCalled = false;
};
\end{lstlisting}

In such case class designer may well not want to give access to
\cppid{alreadyCalled}, since it is only there to help finding bugs. If
\cppid{alreadyCalled} was exposed through a public \cppid{isAlreadyCalled()}
function, users might use it in client code (which seems undesirable for the
original designer). In fact, they should use \cppid{std::call\_once()}.

It is also interesting to note, that if the precondition and the postcondition
where moved into assertions in the body of the function, information would be lost
for some forms of static analysis as well as possible optimizations.

\subsection{Subverting the rules with inheritance}

If a given class has a private member it will not be possible to use that member
in the preconditions of that class. However, a derived class that exposes the
member through \cppkey{using} would be able to use it in its own preconditions.

Additionally, a base class with a private virtual member function may use private
members in its contract conditions. If a derived class overrides the member
function as public we face the paradox that the overridden function needs to have
the same contract conditions but that those conditions cannot be used because
they cannot make use of private members.

\subsection{Violation of information hiding principle}

Some have argued that using private members in the contract condition of a
public function would be leaking implementation details in the interface.
However, it should be noted that we usually have a number of implementation
details in header files today (inline function bodies, private members of
classes or whole implementation of templates).

The rationale behind information hiding principle is to limit the number of
operations that have access to implementation details and to guarantee that
invariants are kept. It should be noted that by allowing the use of private
members in preconditions and postconditions we are not changing anything about
this.

Firstly, the invariants will be still held. This is a consequence from the fact
that any precondition or postcondition that performs and observable modification
of the state of the class would lead to undefined behavior.

Secondly, the client code cannot gain access to private members through the
preconditions or postconditions. The only thing that client code may know is
whether contract conditions are satisfied or not.

\subsection{Expressing semantics accurately}

If a precondition involves private members and we do not want to expose details
by enlarging the class interface, one alternative could be to add an assert to
the function body:

\begin{lstlisting}
class X {
public:
  //...

  // ptr_ must be non-null
  double get_value() const;
private:
  double * ptr_;
};

double X::get_value() const {
  [[assert: ptr_!=nullptr]];
  return ptr_;
}
\end{lstlisting}

However, now it is not so clear that the intent of the developer is that the
client has the responsibility to make sure that the precondition is satisfied
before calling the member function.

\begin{lstlisting}
class X {
public:
  //...

  // ptr_ must be non-null
  double get_value() const [[expects: ptr_!=nullptr]] { return ptr_; }
private:
  double * ptr_;
};
\end{lstlisting}

The use of preconditions (instead of assertions) als makes much more visible
that the function does not have an empty contract and that there are
preconditions to be satisfied.

\subsection{Examples of multistate interfaces}

Mathias Stearn pointed out that
\cppid{std::promise} has many member functions that should only be called if it
is in certain states, but the current state isn’t exposed. For example, you can
only call \cppid{get\_future()} once, and it must not be in a \emph{moved-from}
state. The same thing happens for the completion member functions, as calling any of
them disallows calling any other as well. Now, technically they are defined to
throw if they are misused. Consequently, technically speaking they are not
preconditions and the corresponding member functions have an empty contract (a
contract with now preconditions and no postconditions). 

However, many have argued that those kinds of APIs are a bad idea. In this case
it was because of a potential usage pattern that allows multiple threads to race
to complete the promise with a first-in-wins resolution, but that imposes a huge
cost on all users, when many don’t need that ability, and those that do could
achieve it easily with external synchronization.

Consequently, there is precedent in the standard library for having
preconditions that aren’t queryable and rely on users doing the right thing.

\subsection{Refactoring an utility type}

This idea came up fairly recently in a practical project. The ingredients we have are:

\begin{itemize}

\item
there's a third-party type that looks a lot like a standard type.

\item
there's a desire to migrate the implementation and the uses of that third-party type to the standard type

\item
there are subtle differences in the API semantics of the third-party type and the standard type. What is desired is a way to check that those differences don't cause breakage during migration.

\end{itemize}

Interestingly, we can now say "contracts to the rescue!". So, we have something along the lines of:

\begin{lstlisting}
template <class T>
class MyOptional {
private:
  MyOptional_impl<T> m_impl;
public:
  template <class T>
  MyOptional(T&& val); // constraints omitted

  template <class T>
  MyOptional(const MyOptional<T>&& val);
};
\end{lstlisting}

Now we want to first turn \cppid{MyOptional\_impl} into \cppid{std::optional}, and check that
there was no practical impact. What we'll do is this:

\begin{lstlisting}
template <class T>
class MyOptional {
private:
  MyOptional_impl<T> m_impl;
public:
  template <class U>
  MyOptional(U&& val) // constraints omitted
    [[ensures: different_optionals_initialized_the_same_way()]];

  template <class U>
  MyOptional(const MyOptional<U>& val);
    [[ensures: different_optionals_initialized_the_same_way()]];

  // other ctor overloads omitted

  /* extremely */ private:
  bool different_optionals_initialized_the_same_way();
};
\end{lstlisting}

The idea here is that the check in the postcondition of a constructor verifies
that the initialization of \cppid{MyOptional} did the same thing regardless of whether
\cppid{m\_impl} is \cppid{MyOptional\_impl<T>} or \cppid{std::optional<T>}. The way it does is it
initializes the member as usual, and then in the check, initializes another
object of the different type and verifies that the engagedness and the value
are the same. Same goes for assignments and other operations.

Now, whether other people find such contract uses handy is an interesting
question. To me, this is an astounding capability to have. It makes NO sense to
expose this checking functionality for users of the type, but it's a really
valuable thing to be able to do.


\subsection{Avoiding repetition}

In a function without a body, stating an assertion would imply adding the body
just becase of the assertion. However, any assertion in an empty body may be
seen as a precondition. Consequently, allowing the use of private members in
public functions eliminates a case where otherwise an assertion is needed to
simulate a precondition.

Additionally, in the case of overridden functions, preconditions in the form of
\cppid{[[}\cppkey{expects}\cppid{]]} will be present in all overridden
functions, while ussing assertions to simulate preconditions would lead to the
need of repeating the assertion in each overridden function and the possibility
for forgetting it in some function. 

\subsection{Simplicity and regularity}

Havin special rules for accessibility in preconditions and postconditions are a
complication in the language. Programmers need to remember different sets of
rules when they are writing function bodies or when they are stating
preconditions and postconditions.

A simpler rule is to mandate that the access control performed in preconditions
and postconditions are exactly the same than in function bodies.

\subsection{Supporting multiple styles}

One key characteristic of C++ has been its ability to support multiple styles of
programming and not imposing a single style over the others.

Allowing access to private members while allowed might be rare. Still, some
styles or programming might decide not use private members, while other styles
might need that.

\subsection{How do we handle the problems?}

It is true that in most of the cases preconditions and postconditions should be
expressed in terms of the public interface. However there cases that making of
this an absolute rule would introduce excessive complications. On the other hand
allowing access to private members are a simple solution for those cases.


