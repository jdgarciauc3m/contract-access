\section{Goals}

The proposed change has the following goals:

\begin{enumerate}

\item
Apply contracts to any existing API without changing the (programmatic) API; if
the facilities via which a contract can be checked are not exposed via the
accessible API but are available or injectable via or into the private or
protected API, we still want to be able to apply contracts on such an API,
including preconditions and postconditions.

\item
Design contract-enforced APIs from the scratch without having to introduce API
entry points that expose the correctness checks; such exposure may be
contradictory to the programmatic API's design.

\end{enumerate}

The current access restrictions prevent achieving either of those goals.
Failing to achieve the first goal is a limitation on the applicability of
contracts; some might say there's no harm done and it's a pure opportunity
cost. Failing to achieve the second goal could be said to be even more an
opportunity cost. Those opportunity costs seem to us too high to bear. It seems
plausible that it would be unfortunate indeed if the applicability of contracts
to existing APIs is limited this way, and it seems perhaps even more
unfortunate to bear the cost of not being able to design new contract-enforced
APIs without exposing checking facilities in a programmatic API, because that
seems like a very good use of contracts, and a language capability that wasn't
available before.

Additionally, having different access rules for contracts conditions and all
other code in classes makes the language irregular and more difficult to learn
and teach.
